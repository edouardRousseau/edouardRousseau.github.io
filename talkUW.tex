%%%%%%%%%%%%%%%%%%%%%%%%%%%%%%%%%%%%%%%%%%%%%%%%%%%%%%%%%%%%
%%  This Beamer template was created by Cameron Bracken.
%%  Anyone can freely use or modify it for any purpose
%%  without attribution.
%%
%%  Last Modified: January 9, 2009
%%

\documentclass[xcolor=x11names,compress]{beamer}

%% General document %%%%%%%%%%%%%%%%%%%%%%%%%%%%%%%%%%
\usepackage{graphicx}
\usepackage[utf8]{inputenc}
\usepackage[T1]{fontenc}
\usepackage[english]{babel}
\usepackage{amsmath,amssymb,amsthm,amsopn}
\usepackage{mathrsfs}
\usepackage{graphicx}
\usepackage{tikz}
%\usepackage{array}
%\usepackage[top=1cm,bottom=1cm]{geometry}
%\usepackage{listings}
%\usepackage{xcolor}
\usepackage{hyperref}
\hypersetup{
    colorlinks=true,
    linkcolor=blue,
    citecolor=red,
}

\newtheoremstyle{break}%
{}{}%
{\itshape}{}%
{\bfseries}{}%  % Note that final punctuation is omitted.
{\newline}{}

\newtheoremstyle{sc}%
{}{}%
{}{}%
{\scshape}{}%  % Note that final punctuation is omitted.
{\newline}{}

\theoremstyle{break}
\newtheorem{thm}{Theorem}[section]
\newtheorem{lm}[thm]{Lemma}
\newtheorem{prop}[thm]{Proposition}
\newtheorem{cor}[thm]{Corollary}

\theoremstyle{sc}
\newtheorem{exo}{Exercice}

\theoremstyle{definition}
\newtheorem{defi}[thm]{Definition}
\newtheorem{ex}[thm]{Example}

\theoremstyle{remark}
\newtheorem{rem}[thm]{Remark}

% Raccourcis pour les opérateurs mathématiques (les espaces avant-après sont
% modifiés pour mieux rentrer dans les codes mathématiques usuels)
\DeclareMathOperator{\Ker}{Ker}
\DeclareMathOperator{\Id}{Id}
\DeclareMathOperator{\Img}{Im}
\DeclareMathOperator{\Card}{Card}
\DeclareMathOperator{\Vect}{Vect}
\DeclareMathOperator{\Tr}{Tr}
\DeclareMathOperator{\Mod}{mod}
\DeclareMathOperator{\Ord}{Ord}
\DeclareMathOperator{\ppcm}{ppcm}


% Nouvelles commandes
\newcommand{\ps}[2]{\left\langle#1,#2\right\rangle}
\newcommand{\ent}[2]{[\![#1,#2]\!]}
\newcommand{\diff}{\mathop{}\!\mathrm{d}}
\newcommand{\ie}{\emph{i.e. }}
%%%%%%%%%%%%%%%%%%%%%%%%%%%%%%%%%%%%%%%%%%%%%%%%%%%%%%


%% Beamer Layout %%%%%%%%%%%%%%%%%%%%%%%%%%%%%%%%%%
\useoutertheme[subsection=false,shadow]{miniframes}
\useinnertheme{default}
\usefonttheme{serif}
\usepackage{palatino}

\setbeamerfont{title like}{shape=\scshape}
\setbeamerfont{frametitle}{shape=\scshape}

\setbeamercolor*{lower separation line head}{bg=DeepSkyBlue4} 
% \setbeamercolor*{normal text}{fg=black,bg=white} 
% \setbeamercolor*{alerted text}{fg=red} 
% \setbeamercolor*{example text}{fg=black} 
% \setbeamercolor*{structure}{fg=black} 
%  
\setbeamercolor*{palette tertiary}{fg=black,bg=black!10} 
\setbeamercolor*{palette quaternary}{fg=black,bg=black!10} 
% 
% \renewcommand{\(}{\begin{columns}}
% \renewcommand{\)}{\end{columns}}
% \newcommand{\<}[1]{\begin{column}{#1}}
% \renewcommand{\>}{\end{column}}
%%%%%%%%%%%%%%%%%%%%%%%%%%%%%%%%%%%%%%%%%%%%%%%%%%




\begin{document}


%%%%%%%%%%%%%%%%%%%%%%%%%%%%%%%%%%%%%%%%%%%%%%%%%%%%%%
%%%%%%%%%%%%%%%%%%%%%%%%%%%%%%%%%%%%%%%%%%%%%%%%%%%%%%
\begin{frame}
  \title{Discrete logarithm in finite fields of small characteristic}
%\subtitle{SUBTITLE}
\author{
Édouard Rousseau\\
{\it Université de Versailles\\}
}
\date{\today}
\titlepage
\end{frame}

%%%%%%%%%%%%%%%%%%%%%%%%%%%%%%%%%%%%%%%%%%%%%%%%%%%%%%
%%%%%%%%%%%%%%%%%%%%%%%%%%%%%%%%%%%%%%%%%%%%%%%%%%%%%%
\begin{frame}{Contents}
\tableofcontents
\end{frame}

%%%%%%%%%%%%%%%%%%%%%%%%%%%%%%%%%%%%%%%%%%%%%%%%%%%%%%
%%%%%%%%%%%%%%%%%%%%%%%%%%%%%%%%%%%%%%%%%%%%%%%%%%%%%%
\section{\scshape Introduction}
\subsection{The discrete logarithm problem}
\begin{frame}{Context}
  Let $G$ be a cyclic group generated by an element $g$, we denote by $N$ the
  cardinal of $G$. Then we have a \emph{bijection}
  \[
    \begin{array}{cccc}
      exp_g: & \mathbb{Z}/N\mathbb{Z} & \to & G \\
      & \bar n & \mapsto & g^n
    \end{array}.
  \]

  The inverse of $exp_g$ will be denoted by $log_g$. 
\end{frame}

\begin{frame}{The discrete logarithm problem}
  \begin{itemize}
    \item In practice, given an iteger $n$, we have efficient algorithms
      to compute $g^n$
    \item Given $x = g^n \in G$, we \emph{do not} hace efficient algorithm
      to compute $n$.
  \end{itemize}
  This last problem is called the \emph{discrete logarithm problem}.
\end{frame}

\subsection{Terminology}
\begin{frame}{Definitions}
  To express the hardness of a problem, we study its complexity and we use the
  notation
  \[
    L_N(\alpha, c) = \exp((c+o(1))(\log N)^\alpha(\log\log N)^{1-\alpha}).
  \]
  We also note $L_N(\alpha)$ when we do not want to deal with the constant.

  There are two families of algorithms :
  \begin{itemize}
    \item The \emph{generic} algorithms (with complexity $O(\sqrt N)$)
    \item The \emph{index calculus} algorithms, which use the structure
      of groups coming from finite fields : $\mathbb{F}_{q}^\times$
  \end{itemize}
\end{frame}

\begin{frame}{Definitions}
  \begin{itemize}
    \item We say that a finite field $\mathbb{F}_q=\mathbb{F}_{p^k}$ is of
      \emph{small characteristic} when $p$ is small compared to $q$, it usually
      means that $p$ is polynomial in $\log q$.
    \item Let $l = \log q$, the complexity is then said to be
      \emph{quasi-polynomial} if it is
  $l^{O(\log l)}$. This complexity is smaller that $L(\varepsilon)$ for any
  $\varepsilon>0$ but greater than any polynomial in $l$.
  \end{itemize}
\end{frame}

\subsection{Historical background}
\begin{frame}{Historical background}
  \begin{itemize}
    \item First appearance in [DH76]
    \item First sub-exponential algorithm [A79] : $L(1/2)$
    \item Between 1984 and 2006 : algorithms in $L(1/3)$
\end{itemize}
 And more recently, in finite fields of small characteristic :
  \begin{itemize}
    \item New algorithm with $L(1/4)$ complexity [Joux13]
    \item \emph{Quasi-polynomial} algorithm [BGJT14]
    \item Second quasi-polynomial algorithm [GKZ14]
  \end{itemize}
\end{frame}

\section{\scshape Index calculus}
\subsection{Overview}
\begin{frame}{Overview}
  Assume that we want to find $log_g(h)$ with $h\in G$. We fist choose $F\subset
  G$ such that $\left\langle F \right\rangle = G$. Then
  \begin{enumerate}
    \item We find multiplicative relations between the elements in $F$
    \item We solve the linear system arising from these relations
    \item We express $h$ as a product of elements in $F$
  \end{enumerate}
  The steps $1$ and $3$ depends on the representation of the finite field, and
  give different complexities.
\end{frame}

\subsection{An example}
\begin{frame}{An example}
  \emph{Context :}
  \begin{itemize}
    \item We considere $G = \mathbb{F}_p^\times$ for a prime integer $p$ and we
      still have $N = |G|$
    \item We choose $F = \left\{\, f \;|\; f \leq B,\; f \text{ prime}
    \right\}$ for a chosen integer $B$
  \item We assume that $g\in F$, otherwise we add it to $F$.
  \end{itemize}
\end{frame}

\begin{frame}{An example}
  \emph{Step 1 : relations generation}
  \begin{itemize}
    \item We randomly choose $e\in \mathbb{Z}/N\mathbb{Z}$
    \item We compute $g^e$
    \item We test if $g^e$, seen as an integer, is $B$-smooth, \ie has only
      prime factors $\leq B$
    \item If it is the case, it yields a relation in $G$ :
      \[ 
        g^e = \prod_{f\in F}f^{e_f}, \text{ où } e_f\in \mathbb{N}
      \]
      that can be written
      \[
        e = \sum_{f\in F}e_f\log_g(f).
      \]
  \end{itemize}
\end{frame}

\begin{frame}{An example}
  \emph{Step 2 : linear algebra.} Once we have enough relations, \ie at least
  $|F|$, we solve the linear system in $\mathbb{Z}/N\mathbb{Z}$ in order to
  obtain the $\log_g(f)$ for all $f\in F$.
  
  \emph{Step 3 : express $h$ in function of the elements in $F$ :}
  \begin{itemize}
     \item We randomly choose $e\in \mathbb{Z}/N\mathbb{Z}$
    \item We compute $hg^e$
    \item We test if $hg^e$ is $B$-smooth
    \item If it is the case, it yields a relation  :
     \[
      \log_g(h) = \sum_{f\in F}e_f\log_g(f) - e
      \]
  \end{itemize}

\end{frame}

\begin{frame}{An example}
  This algorithm depends on the choice of $B$ :
  \begin{itemize}
    \item The larger $B$ is, the larger $\left\langle F \right\rangle$ is, and
      the easier it is to find relations
    \item But when $|F|$ is large, we need to solve a large linear system
  \end{itemize}
  In the end, we can choose $B$ to obtain a $L(1/2)$ complexity.
\end{frame}

\section{\scshape Quasi-polynomial algorithms} 
\subsection{Barbulescu, Gaudry, Joux and Thomé algorithm}

\begin{frame}{Barbulescu, Gaudry, Joux et Thomé algorithm}
  \emph{Context :} we denote by $\mathbb{K}$ the finite field where we want to
  compute discrete logarithms. 
  \begin{itemize}
    \item We assume that $\mathbb{K}=\mathbb{F}_{q^{2k}}$ and we represent
      $\mathbb{K}$ by $\mathbb{F}_{q^2}[X]/(I_X)$ where $I_X$ is an irreducible
      polynomial of degree $k$ dividing $h_1X^q-h_0$ and $\deg h_i\leq 2$.
      \begin{itemize}
        \item The existence of suitable $h_i$ is heuristic
      \end{itemize}
    \item The set $F$ is the set of the degree one polynomials.
\end{itemize}
\end{frame}

\begin{frame}{Barbulescu, Gaudry, Joux et Thomé algorithm}
The algorithm is based on a descent process: we express the logarithm of a
polynomial as a linear combination of $O(q^2k)$ logarithms of polynomials of degree two
times smaller, until we have only polynomials in $F$.
    \begin{itemize}
      \item Complexity : $(q^2k)^{O(\log k)}$.
    \end{itemize}
    The descent process is based on the equation :
 \begin{equation}
   X^qY - XY^q = Y\prod_{a\in\mathbb{F}_q}(X - aY) =
   \prod_{\alpha\in\mathbb{P}^1(\mathbb{F}_q)}(X-\alpha Y)
   \label{keyeq}
 \end{equation}
 
\end{frame}

\begin{frame}{The descent}
  Assume that we want to find the logarithm of an element $P$. We will create
  relations by substituting $X$ by $aP + b$ and $Y$ by $cP + d$ in Equation
  (\ref{keyeq}), with $a, b, c, d \in \mathbb{F}_{q^2}$. We obtain a new
  Equation $(E_{a, b, c, d})$. It follows that
  \begin{eqnarray*}
    \frac{1}{h_1^D}\mathcal L_{a, b, c, d} &=& \lambda
    \prod_{\alpha\in\mathbb{P}^1(\mathbb{F}_q)}(P-\mu_\alpha)
  \end{eqnarray*}
  where $\lambda, \mu_\alpha\in\mathbb{F}_{q^2}$ and $\mathcal L_{a, b, c,
  d}$ is a polynomial of degree $D\leq 3\deg P$, obtained using the equality $X^q =
  \frac{h_0}{h_1}\mod I_X$.
  
\end{frame}

\begin{frame}{The descent}
  We keep only the equations $(E_{a, b, c, d})$ where
  $\mathcal L_{a, b, c, d}$ is
  $\left\lceil \frac{\deg P}{2}\right\rceil$-smooth and we combine these equations
  in order to keep only $P$ in the right hand side.

  The left hand side is then composed of irreducible polynomials 
  of degree at most $\left\lceil \frac{\deg P}{2}\right\rceil$.

  \begin{itemize}
    \item There are also heuristics
      \begin{itemize}
        \item The existence of the combination
        \item The smoothness of the polynomials $\mathcal L_{a, b, c, d}$
      \end{itemize}
  \end{itemize}
\end{frame}


\subsection{Granger, Kleinjung and Zumbrägel algorithme}

\begin{frame}{Granger, Kleinjung and Zumbrägel algorithm}
  \emph{Context :} here $\mathbb{K}=\mathbb{F}_{q^k}$ and we see $\mathbb{K}$ as
  $\mathbb{F}_q[X]/(I_X)$ where $I_X$ is a polynomial of degree $k$ dividing
  $h_1X^q-h_0$. The algorithm follows the same steps as the latter, but the
  descent is different.
\end{frame}

\begin{frame}{``On the fly'' elimination}
  Assume that $Q\in \mathbb{F}_{q^m}[X]$ and $\deg Q = 2$. This elimination is
  based on the fact that the polynomial $P = X^{q+1}+aX^q+bX+c$ splits completly
  in $\mathbb{F}_{q^m}[X]$ for approximately $q^{m-3}$ triplets $(a, b, c)$. But
  
  \[
    P = \frac{1}{h_1}((X+a)h_0 + (bX+c)h_1)\mod I_X
  \]
  So if $Q| (X+a)h_0 + (bX+c)h_1$ (polynomial of degree $3$), we have 
  \[
    h_1P=QL \mod I_X
  \]
  where $L$ is of degree $1$ and $P$ splits completly.

\end{frame}

\begin{frame}{The descent}
  Assume now that $Q\in\mathbb{F}_{q}[X]$ is irreducible of degree $2d$. Then we
  have
  \[
    Q=\prod_{i=0}^{d-1} Q_i = \prod_{i=0}^{d-1}Q_0^{[i]}
  \]
  where the $Q_i$'s are irreducible polynomials of degree $2$ in
  $\mathbb{F}_{q^d}[X]$ and are all conjugates, in the sense that
  $Q_i=Q_0^{[i]}$ denotes the polynomial obtained by raising all coefficients to
  the power $q^i$.
\end{frame}
  
\begin{frame}{The descent}
  \begin{itemize}
    \item We then apply the ``on the fly''
      elimination to $Q_0^{[i]}$ to obtain $Q_0^{[i]}$ as a product of 
      $O(q)$ $P_j^{[i]}$ where the $P_j\in \mathbb{F}_{q^d}[X]$ are all of degree $1$
    \item Hence, $Q$ is expressed as a product of $O(q)$ norms of linear polynomial
      $\prod_{i=0}^{d-1}P_j^{[i]}$
  \end{itemize}
  
  Recall that the norm of a linear polynomial in $\mathbb{F}_{q^d}[X]$ is an irreducible
  polynomial of degree $d_1$ to the power $d_2$, with $d_1d_2 = d$.
  \begin{itemize}
    \item \emph{Thus, we expressed $\log Q$ as a linear combination of $O(q)$ $\log
        R_i$ where $\deg R_i | d$. The complexity obtained is $q^{O(\log q)}$}.
  \end{itemize}
\end{frame}

%%%%%%%%%%%%%%%%%%%%%%%%%%%%%%%%%%%%%%%%%%%%%%%%%%%%%%
%%%%%%%%%%%%%%%%%%%%%%%%%%%%%%%%%%%%%%%%%%%%%%%%%%%%%%
\end{document}
