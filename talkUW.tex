%%%%%%%%%%%%%%%%%%%%%%%%%%%%%%%%%%%%%%%%%%%%%%%%%%%%%%%%%%%%
%%  This Beamer template was created by Cameron Bracken.
%%  Anyone can freely use or modify it for any purpose
%%  without attribution.
%%
%%  Last Modified: January 9, 2009
%%

\documentclass[xcolor=x11names,compress]{beamer}

%% General document %%%%%%%%%%%%%%%%%%%%%%%%%%%%%%%%%%
\usepackage{graphicx}
\usepackage[utf8]{inputenc}
\usepackage[T1]{fontenc}
\usepackage[english]{babel}
\usepackage{amsmath,amssymb,amsthm,amsopn}
\usepackage{mathrsfs}
\usepackage{graphicx}
\usepackage{tikz}
%\usepackage{array}
%\usepackage[top=1cm,bottom=1cm]{geometry}
%\usepackage{listings}
%\usepackage{xcolor}
\usepackage{hyperref}
\hypersetup{
    colorlinks=true,
    linkcolor=blue,
    citecolor=red,
}

\newtheoremstyle{break}%
{}{}%
{\itshape}{}%
{\bfseries}{}%  % Note that final punctuation is omitted.
{\newline}{}

\newtheoremstyle{sc}%
{}{}%
{}{}%
{\scshape}{}%  % Note that final punctuation is omitted.
{\newline}{}

\theoremstyle{break}
\newtheorem{thm}{Theorem}[section]
\newtheorem{lm}[thm]{Lemma}
\newtheorem{prop}[thm]{Proposition}
\newtheorem{cor}[thm]{Corollary}

\theoremstyle{sc}
\newtheorem{exo}{Exercice}

\theoremstyle{definition}
\newtheorem{defi}[thm]{Definition}
\newtheorem{ex}[thm]{Example}

\theoremstyle{remark}
\newtheorem{rem}[thm]{Remark}

% Raccourcis pour les opérateurs mathématiques (les espaces avant-après sont
% modifiés pour mieux rentrer dans les codes mathématiques usuels)
\DeclareMathOperator{\Ker}{Ker}
\DeclareMathOperator{\Id}{Id}
\DeclareMathOperator{\Img}{Im}
\DeclareMathOperator{\Card}{Card}
\DeclareMathOperator{\Vect}{Vect}
\DeclareMathOperator{\Tr}{Tr}
\DeclareMathOperator{\Mod}{mod}
\DeclareMathOperator{\Ord}{Ord}
\DeclareMathOperator{\ppcm}{ppcm}


% Nouvelles commandes
\newcommand{\ps}[2]{\left\langle#1,#2\right\rangle}
\newcommand{\ent}[2]{[\![#1,#2]\!]}
\newcommand{\diff}{\mathop{}\!\mathrm{d}}
\newcommand{\ie}{\emph{i.e. }}
\newcommand{\eg}{\emph{e.g. }}
%%%%%%%%%%%%%%%%%%%%%%%%%%%%%%%%%%%%%%%%%%%%%%%%%%%%%%


%% Beamer Layout %%%%%%%%%%%%%%%%%%%%%%%%%%%%%%%%%%
\useoutertheme[subsection=false,shadow]{miniframes}
\useinnertheme{default}
\usefonttheme{serif}
\usepackage{palatino}

\setbeamerfont{title like}{shape=\scshape}
\setbeamerfont{frametitle}{shape=\scshape}

\setbeamercolor*{lower separation line head}{bg=DeepSkyBlue4} 
% \setbeamercolor*{normal text}{fg=black,bg=white} 
% \setbeamercolor*{alerted text}{fg=red} 
% \setbeamercolor*{example text}{fg=black} 
% \setbeamercolor*{structure}{fg=black} 
%  
\setbeamercolor*{palette tertiary}{fg=black,bg=black!10} 
\setbeamercolor*{palette quaternary}{fg=black,bg=black!10} 
% 
% \renewcommand{\(}{\begin{columns}}
% \renewcommand{\)}{\end{columns}}
% \newcommand{\<}[1]{\begin{column}{#1}}
% \renewcommand{\>}{\end{column}}
%%%%%%%%%%%%%%%%%%%%%%%%%%%%%%%%%%%%%%%%%%%%%%%%%%




\begin{document}


%%%%%%%%%%%%%%%%%%%%%%%%%%%%%%%%%%%%%%%%%%%%%%%%%%%%%%
%%%%%%%%%%%%%%%%%%%%%%%%%%%%%%%%%%%%%%%%%%%%%%%%%%%%%%
\begin{frame}
  \title{Discrete logarithm in finite fields of small characteristic}
%\subtitle{SUBTITLE}
\author{
Édouard Rousseau\\
{\it Université de Versailles\\}
}
\date{\today}
\titlepage
\end{frame}

%%%%%%%%%%%%%%%%%%%%%%%%%%%%%%%%%%%%%%%%%%%%%%%%%%%%%%
%%%%%%%%%%%%%%%%%%%%%%%%%%%%%%%%%%%%%%%%%%%%%%%%%%%%%%
\begin{frame}{Contents}
\tableofcontents
\end{frame}

%%%%%%%%%%%%%%%%%%%%%%%%%%%%%%%%%%%%%%%%%%%%%%%%%%%%%%
%%%%%%%%%%%%%%%%%%%%%%%%%%%%%%%%%%%%%%%%%%%%%%%%%%%%%%
\section{\scshape Introduction}
\subsection{The discrete logarithm problem}
\begin{frame}{Context}
  \begin{itemize}
    \item $G=\left\langle g \right\rangle$ cyclic group generated by $g$
    \item $N=\Card G$.
    \end{itemize}
  
  We have a \emph{bijection}
  \[
    \begin{array}{cccc}
      \exp_g: & \mathbb{Z}/N\mathbb{Z} & \to & G \\
      & \bar n & \mapsto & g^n
    \end{array}
  \]

  \begin{itemize}
    \item $\log_g:=\exp_g^{-1}$
   \item Compute $g^n$ from $n$ : \textcolor{green}{\bf easy}
    \item Compute $n$ from $g^n$ : \textcolor{red}{\textbf{hard} (discrete
    logarithm problem)}
  \end{itemize}
\end{frame}

\subsection{Terminology}
\begin{frame}{Definitions}
  Notation :
  \[
    \textcolor{purple}{L_N(\alpha) = \exp((c+o(1))(\log N)^\alpha(\log\log
      N)^{1-\alpha})}.
  \]

  Two families of algorithms :
  \begin{itemize}
    \item The \emph{generic} algorithms (complexity :
      $\textcolor{purple}{O(\sqrt N)}$)
    \item The \emph{index calculus} algorithms, using group structure
    \end{itemize}
    Terminology :
    \begin{itemize}
    \item \emph{small characteristic} : $\mathbb{F}_q$ with $q=p^k$ and $p\ll q$
    \item \emph{quasi-polynomial} complexity : $\textcolor{purple}{\log
        q^{O(\log \log q)}}$
  \end{itemize}
\end{frame}

\subsection{Historical background}
\begin{frame}{Historical background}
  \begin{itemize}
    \item First appearance in \textcolor{blue}{[DH76]}
    \item First sub-exponential algorithm \textcolor{blue}{[A79]} :
      $\textcolor{purple}{L(1/2)}$
    \item Between 1984 and 2006 : algorithms in $\textcolor{purple}{L(1/3)}$
\end{itemize}
 And more recently, in finite fields of small characteristic :
  \begin{itemize}
    \item New algorithm with $\textcolor{purple}{L(1/4)}$ complexity
      \textcolor{blue}{[Joux13]}
    \item \emph{Quasi-polynomial} algorithm \textcolor{blue}{[BGJT14]}
    \item Second quasi-polynomial algorithm \textcolor{blue}{[GKZ14]}
  \end{itemize}
\end{frame}

\section{\scshape Index calculus}
\subsection{Overview}
\begin{frame}{Overview}
  \emph{Goal :} find $log_g(h)$ 
  \begin{enumerate}
    \item[0.] first choose $F\subset
  G$ with $\left\langle F \right\rangle = G$
    \item find multiplicative relations between elements in $F$
    \item solve the associated linear system 
    \item express $h$ as a product of elements in $F$
  \end{enumerate}
  The steps $1$ and $3$ depends on the representation of the finite field, and
  give different complexities.
\end{frame}

\subsection{An example}
\begin{frame}{An example}
  \emph{Context :}
  \begin{itemize}
    \item $G = \mathbb{F}_p^\times$ for a prime $p$ and $N = |G|$
    \item $F = \left\{\, f \;|\; f \leq B,\; f \text{ prime}
    \right\}$ for a chosen integer $B$
  \end{itemize}

  \emph{Step 1 : relations generation}
  \begin{itemize}
    \item randomly choose $e\in \mathbb{Z}/N\mathbb{Z}$
    \item test if $g^e$ is $B$-smooth
    \item if it is the case, it yields a relation in $G$ :
      \[ 
        g^e = \prod_{f\in F}f^{e_f}, \text{ où } e_f\in \mathbb{N}
      \]
      that can be written
      \[
        e = \sum_{f\in F}e_f\log_g(f).
      \]
  \end{itemize}
\end{frame}

\begin{frame}{An example}
  \emph{Step 2 : linear algebra :} solve the linear system
  
  \emph{Step 3 : express $h$ in function of the elements in $F$ :}
  \begin{itemize}
     \item randomly choose $e\in \mathbb{Z}/N\mathbb{Z}$
    \item test if $hg^e$ is $B$-smooth
    \item if it is the case, it yields a relation  :
     \[
      \log_g(h) = \sum_{f\in F}e_f\log_g(f) - e
      \]
  \end{itemize}

 Depends on $B$ :
  \begin{itemize}
    \item $B$ large : easier to find relations
    \item $B$ large : need more relations to solve the system
  \end{itemize}
  Complexity : $\textcolor{purple}{L(1/2)}$

\end{frame}

\section{\scshape Quasi-polynomial algorithms} 
\subsection{Barbulescu, Gaudry, Joux and Thomé algorithm}

\begin{frame}{Barbulescu, Gaudry, Joux and Thomé algorithm}
  \emph{Context :}
  \begin{itemize}
    \item $\mathbb{K}=\mathbb{F}_{q^{2k}}=\mathbb{F}_{q^2}[X]/(I_X)$ with $I_X$ irreducible
      polynomial of degree $k$ dividing $h_1X^q-h_0$ and $\deg h_i\leq 2$.
      \begin{itemize}
        \item \textcolor{violet}{\textbf{heuristic :}} existence of $h_i$
      \end{itemize}
    \item $F=\left\{  \text{degree one polynomials}\right\}$ 
\end{itemize}
\emph{Idea:} \textcolor{violet}{\textbf{descent}} process:
  \[
    P \leadsto O(q^2k)\; P_j   
  \]
    
    with $\deg P_j \leq \left\lceil\frac{1}{2}\deg P\right\rceil$

    \begin{itemize}
      \item Complexity : $\textcolor{purple}{(q^2k)^{O(\log k)}}$.
    \end{itemize}

\end{frame}

\begin{frame}{The descent}
    Based on the equation : 
 \begin{equation}
   X^qY - XY^q = Y\prod_{a\in\mathbb{F}_q}(X - aY) =
   \prod_{\alpha\in\mathbb{P}^1(\mathbb{F}_q)}(X-\alpha Y)
   \label{keyeq}
 \end{equation}
 
 \begin{itemize}
   \item Substitute $X$ by $aP + b$ and $Y$ by $cP + d$ in (\ref{keyeq}), with
     $a, b, c, d \in \mathbb{F}_{q^2}$, use $X^q =\frac{h_0}{h_1}\mod I_X$ : 

 \end{itemize}
  \begin{eqnarray*}
    \frac{1}{h_1^D}\mathcal L_{a, b, c, d} &=& \lambda
    \prod_{\alpha\in\mathbb{P}^1(\mathbb{F}_q)}(P-\mu_\alpha)
  \end{eqnarray*}
  where $\lambda, \mu_\alpha\in\mathbb{F}_{q^2}$ and $\deg\mathcal L_{a, b, c,
  d}\leq 3\deg P$   
\end{frame}

\begin{frame}{The descent}
  \begin{itemize}
    \item keep only the equations $(E_{a, b, c, d})$ where $\mathcal L_{a, b, c, d}$ is
  $\left\lceil \frac{\deg P}{2}\right\rceil$-smooth
\item combine these equations to keep only $P$ in the right hand side.
\item left hand side : irreducible polynomials 
  of degree~$\leq\left\lceil \frac{\deg P}{2}\right\rceil$.
  \end{itemize}


  \begin{itemize}
    \item \textcolor{violet}{\textbf{heuristics :}}
      \begin{itemize}
        \item The existence of the combination
        \item The smoothness of the polynomials $\mathcal L_{a, b, c, d}$
      \end{itemize}
  \end{itemize}
\end{frame}


\subsection{Granger, Kleinjung and Zumbrägel algorithm}

\begin{frame}{Granger, Kleinjung and Zumbrägel algorithm}
  \emph{Context:}
  \begin{itemize}
    \item $\mathbb{K}=\mathbb{F}_{q^k} = \mathbb{F}_q[X]/(I_X)$ with
  $I_X$ irreducible polynomial of degree $k$ dividing $h_1X^q-h_0$.
  \end{itemize}
  \emph{Ideas:}
  \begin{itemize}
    \item \textcolor{violet}{\textbf{descent}} process
    \item ``on the fly'' elimination
  \end{itemize}
\end{frame}

\begin{frame}{``On the fly'' elimination}
  \emph{Input:}
  \begin{itemize}
    \item $Q\in \mathbb{F}_{q^m}[X]$ and $\deg Q = 2$
  \end{itemize}
  \emph{Output:}
  \begin{itemize}
    \item $Q\leadsto O(q)\; Q_i$ with $\deg Q_i = 1$ and $Q_i\in
      \mathbb{F}_{q^m}[X]$
  \end{itemize}

  \emph{Ideas:}
  \begin{itemize}
    \item $P = X^{q+1}+aX^q+bX+c$ splits completly
  in $\mathbb{F}_{q^m}[X]$ for  $\approx q^{m-3}$ triples $(a, b, c)$.
  \end{itemize}
\end{frame}

\begin{frame}{``On the fly'' elimination}
  \[
    P = \frac{1}{h_1}((X+a)h_0 + (bX+c)h_1)\mod I_X
  \]
  \begin{itemize}
    \item if $Q| (X+a)h_0 + (bX+c)h_1$ (degree $3$), we have 
  \end{itemize}
  \[
    h_1P=QL \mod I_X
  \]
  where $L$ is of degree $1$ and $P$ splits completly.

\end{frame}

\begin{frame}{The descent}
  $Q\in\mathbb{F}_{q}[X]$ irreducible of degree $2d$, we have
  \[
    Q=\prod_{i=0}^{d-1} Q_i = \prod_{i=0}^{d-1}Q_0^{[i]}
  \]
  with 
  \begin{itemize}
    \item $Q_i$ irreducible of degree $2$ in $\mathbb{F}_{q^d}[X]$ 
    \item $Q_i=Q_0^{[i]}$ conjugate of $Q_0$
  \end{itemize}
\end{frame}
  
\begin{frame}{The descent}
  \begin{itemize}
    \item ``on the fly'' elimination: $Q_0^{[i]}\leadsto O(q)\;P_j^{[i]}$
    \item $Q\leadsto O(q)\;N_j$ with $N_j=\prod_{i=0}^{d-1}P_j^{[i]}$
  \end{itemize}
  
  $N_j = R_j^{e_{j}}$ with $R_j\in \mathbb{F}_q[X]$ irreducible of degree
  $f_{j}$ and $e_{j}f_{j} = d$
  \begin{itemize}
    \item $Q\leadsto O(q)\;R_j$ with $\deg R_j \;|\;d$
    \item complexity: $\textcolor{purple}{q^{O(\log q)}}$.
  \end{itemize}
\end{frame}

%%%%%%%%%%%%%%%%%%%%%%%%%%%%%%%%%%%%%%%%%%%%%%%%%%%%%%
%%%%%%%%%%%%%%%%%%%%%%%%%%%%%%%%%%%%%%%%%%%%%%%%%%%%%%
\end{document}
