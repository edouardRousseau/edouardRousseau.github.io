%%%%%%%%%%%%%%%%%%%%%%%%%%%%%%%%%%%%%%%%%%%%%%%%%%%%%%%%%%%%
%%  This Beamer template was created by Cameron Bracken.
%%  Anyone can freely use or modify it for any purpose
%%  without attribution.
%%
%%  Last Modified: January 9, 2009
%%

\documentclass[xcolor=x11names,compress]{beamer}

%% General document %%%%%%%%%%%%%%%%%%%%%%%%%%%%%%%%%%
\usepackage{graphicx}
\usepackage[utf8]{inputenc}
\usepackage[T1]{fontenc}
\usepackage[french]{babel}
\usepackage{amsmath,amssymb,amsthm,amsopn}
\usepackage{mathrsfs}
\usepackage{graphicx}
\usepackage{tikz}
%\usepackage{array}
%\usepackage[top=1cm,bottom=1cm]{geometry}
%\usepackage{listings}
%\usepackage{xcolor}
\usepackage{hyperref}
\hypersetup{
    colorlinks=true,
    linkcolor=blue,
    citecolor=red,
}

\newtheoremstyle{break}%
{}{}%
{\itshape}{}%
{\bfseries}{}%  % Note that final punctuation is omitted.
{\newline}{}

\newtheoremstyle{sc}%
{}{}%
{}{}%
{\scshape}{}%  % Note that final punctuation is omitted.
{\newline}{}

\theoremstyle{break}
\newtheorem{thm}{Theorem}[section]
\newtheorem{lm}[thm]{Lemma}
\newtheorem{prop}[thm]{Proposition}
\newtheorem{cor}[thm]{Corollary}

\theoremstyle{sc}
\newtheorem{exo}{Exercice}

\theoremstyle{definition}
\newtheorem{defi}[thm]{Definition}
\newtheorem{ex}[thm]{Example}

\theoremstyle{remark}
\newtheorem{rem}[thm]{Remark}

% Raccourcis pour les opérateurs mathématiques (les espaces avant-après sont
% modifiés pour mieux rentrer dans les codes mathématiques usuels)
\DeclareMathOperator{\Ker}{Ker}
\DeclareMathOperator{\Id}{Id}
\DeclareMathOperator{\Img}{Im}
\DeclareMathOperator{\Card}{Card}
\DeclareMathOperator{\Vect}{Vect}
\DeclareMathOperator{\Tr}{Tr}
\DeclareMathOperator{\Mod}{mod}
\DeclareMathOperator{\Ord}{Ord}
\DeclareMathOperator{\ppcm}{ppcm}


% Nouvelles commandes
\newcommand{\ps}[2]{\left\langle#1,#2\right\rangle}
\newcommand{\ent}[2]{[\![#1,#2]\!]}
\newcommand{\diff}{\mathop{}\!\mathrm{d}}
\newcommand{\ie}{\emph{i.e. }}
%%%%%%%%%%%%%%%%%%%%%%%%%%%%%%%%%%%%%%%%%%%%%%%%%%%%%%


%% Beamer Layout %%%%%%%%%%%%%%%%%%%%%%%%%%%%%%%%%%
\useoutertheme[subsection=false,shadow]{miniframes}
\useinnertheme{default}
\usefonttheme{serif}
\usepackage{palatino}

\setbeamerfont{title like}{shape=\scshape}
\setbeamerfont{frametitle}{shape=\scshape}

\setbeamercolor*{lower separation line head}{bg=DeepSkyBlue4} 
% \setbeamercolor*{normal text}{fg=black,bg=white} 
% \setbeamercolor*{alerted text}{fg=red} 
% \setbeamercolor*{example text}{fg=black} 
% \setbeamercolor*{structure}{fg=black} 
%  
\setbeamercolor*{palette tertiary}{fg=black,bg=black!10} 
\setbeamercolor*{palette quaternary}{fg=black,bg=black!10} 
% 
% \renewcommand{\(}{\begin{columns}}
% \renewcommand{\)}{\end{columns}}
% \newcommand{\<}[1]{\begin{column}{#1}}
% \renewcommand{\>}{\end{column}}
%%%%%%%%%%%%%%%%%%%%%%%%%%%%%%%%%%%%%%%%%%%%%%%%%%




\begin{document}


%%%%%%%%%%%%%%%%%%%%%%%%%%%%%%%%%%%%%%%%%%%%%%%%%%%%%%
%%%%%%%%%%%%%%%%%%%%%%%%%%%%%%%%%%%%%%%%%%%%%%%%%%%%%%
\begin{frame}
  \title{Logarithme discret dans les corps finis de petite charactéristique}
%\subtitle{SUBTITLE}
\author{
Édouard Rousseau\\
{\it Université de Versailles\\}
}
\date{\today}
\titlepage
\end{frame}

%%%%%%%%%%%%%%%%%%%%%%%%%%%%%%%%%%%%%%%%%%%%%%%%%%%%%%
%%%%%%%%%%%%%%%%%%%%%%%%%%%%%%%%%%%%%%%%%%%%%%%%%%%%%%
\begin{frame}{Table des matières}
\tableofcontents
\end{frame}

%%%%%%%%%%%%%%%%%%%%%%%%%%%%%%%%%%%%%%%%%%%%%%%%%%%%%%
%%%%%%%%%%%%%%%%%%%%%%%%%%%%%%%%%%%%%%%%%%%%%%%%%%%%%%
\section{\scshape Introduction}
\subsection{Le problème du logarithme discret}
\begin{frame}{Contexte}
  Soit $G$ un groupe cyclique engendré par un élément $g$, et notons $N$ le
  cardinal de $G$. On a alors une \emph{bijection}
  \[
    \begin{array}{cccc}
      exp_g: & \mathbb{Z}/N\mathbb{Z} & \to & G \\
      & \bar n & \mapsto & g^n
    \end{array}.
  \]

  On note la bijection récriproque $log_g$. 
\end{frame}

\begin{frame}{Le problème du logarithme discret}
  \begin{itemize}
    \item En pratique, étant donné un entier $n$, on dispose d'algorithmes efficaces
  pour calculer $g^n$.
\item Étant donné $x = g^n\in G$, \emph{on ne dispose pas} d'algorithmes efficaces
  pour retrouver $n$.
  \end{itemize}
  C'est ce problème, « Comment retrouver $n$ ? », qu'on appelle \emph{Problème
  du logarithme discret}.
\end{frame}

\begin{frame}{Intérêt en cryptographie}
  En cryptographie, $exp_g$ est appelée \emph{fonction à sens unique} car
  \begin{itemize}
    \item $exp_g$ est facile à calculer
    \item sa réciproque, $log_g$, est difficile à calculer.
  \end{itemize}

  Ces fonctions sont très étudiées en cryptographie car elles permettent
  typiquement de rendre l'opération de cryptage simple et l'opération de
  décryptage longue et compliquée.
\end{frame}

\subsection{Historique}
\begin{frame}{Bref historique du problème}
  Pour exprimer la difficulté d'un problème, on parle de sa complexité et on
  utilise la notation 
  \[
    L_N(\alpha, c) = \exp((c+o(1))(\log N)^\alpha(\log\log N)^{1-\alpha}).
  \]
  On note aussi $L_N(\alpha)$ quand on ne veut pas préciser la constante.

  On distingue deux types d'algorithmes :
  \begin{itemize}
    \item Les algorithmes \emph{génériques} en $O(\sqrt N)$
    \item Les algorithmes de \emph{calcul d'indice}, qui exploitent la structure de
      groupes issus de corps finis : $\mathbb{F}_{q}^\times$
  \end{itemize}
\end{frame}

\begin{frame}{Bref historique du problème}
  \begin{itemize}
    \item Apparition dans ``New directions in cryptography'' de Diffie et
      Hellman (1976)
    \item Premier algorithme sous-exponentiel : Adleman (1979), complexité en
      $L_N(1/2)$
  \item Premier algorithme en $L(1/3)$ dans les corps finis binaires : Coppersmith (1984)
  \item Crible algébrique dans les corps premiers : 1993,
    $L(1/3)$
  \item Crible géométrique, utilisable en petite caractéristique : 1994,
    généralisations en 1999, 2002, 2006; $L(1/3)$
  \item Généralisation du crible algébrique en 2006, on peut dès lors résoudre
    des logarithmes discret en $L(1/3)$ dans tout type de corps fini
  \end{itemize}
\end{frame}

\begin{frame}{Bref historique du problème}
  Et plus récemment, dans les corps finis de petite caractéristique :
  \begin{itemize}
    \item Nouvel algorithme dû à Joux (2013) en $L(1/4)$
    \item Algorithme \emph{quasi-polynomial} dû à Barbulescu, Gaudry, Joux et
      Thomé en 2014
    \item Second algorithme quasi-polynomial dû à Granger, Kleinjung et
      Zumbrägel en 2014
  \end{itemize}
\end{frame}

\section{\scshape Calcul d'indice}
\begin{frame}{Principe général}
  \begin{itemize}
    \item bla
  \end{itemize}
\end{frame}

\section{\scshape Algorithmes quasi-polynomiaux}
\begin{frame}{Définition}
 Bla 
\end{frame}
%%%%%%%%%%%%%%%%%%%%%%%%%%%%%%%%%%%%%%%%%%%%%%%%%%%%%%
%%%%%%%%%%%%%%%%%%%%%%%%%%%%%%%%%%%%%%%%%%%%%%%%%%%%%%
\end{document}
